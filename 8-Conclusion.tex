\chapter{Conclusion} \label{cha:Conclusion}

In any complex, non-convex optimisation problem the only way to be certain of an optimal solution is a brute force search. This is of course computationally prohibitive. Luckily, in the context of planetary missions it is not an operational requirement to find the optimal path, merely to find a good path. In this thesis a procedure to find good paths has been outlined for \BW, and recommendations are made for improving the procedure in the future.

% Findings
\section{Summary of major findings}

A complete trajectory has been presented for each phase in the mission architecture, with a coasting phase inserted between cruise and capture. To achieve the phase boundaries defined in the mission architecture, the transfer from GTO to LLO requires a total of 4.2~kms$^{-1}$. Chemically propelled trajectories typically required about 4~kms$^{-1}$ from LEO to LLO, but as most of this impulse is applied deep within the Earth's gravity well it is magnified by the Oberth effect.

The ascent through the van Allen belts uses the higher-thrust, lower efficiency thermal arcjet to minimise the time spent in this damaging environment. Following a simple tangential thrust scheme to raise periapsis as quickly as possible resulted in this phase taking 32.22~days. A fuel optimal trajectory required 34.17~days but saved over 6~kg of propellant, which is 10\% of the fuel required for the phase. The remaining numerical results are summarised in \autoref{tab:Results}.

\begin{table}
\caption{Summary of numerical results from trajectory simulation}
\label{tab:Results}
\centering
\begin{tabular}{ccccc} \toprule
Phase & Time & Propellant & Delta-v & Energy \tabularnewline
& (days) & (kg) & (kms$^{-1}$) & (MWh) \tabularnewline\midrule
Ascent & 32.22 & 59.82 & 1.3 & 1.14 \tabularnewline
Cruise & 1123.91 & 17.55 & 2.5 & 32.18 \tabularnewline
Propagate & 30.0 & 0.0 & 0.0 & 1.15 \tabularnewline
Capture &18.0 & 33.43 & 1.0 & 0.475 \tabularnewline
Descent & 91.79 & 1.43 & 0.4 & 2.72 \tabularnewline
Science &180.0 & 0.0 & 0.0 & 5.50 \tabularnewline\bottomrule
\end{tabular}
\end{table}

Based on this simulation the arcjet requires a total of 1205.28~hours of operation, and the PPTs require 29176.82~hours of operation. For comparison, the Hall-effect thrusters used for 4958~hours during the SMART-1 mission had a total of 9200~hours of ground qualification testing before flight. Maintaining the same ratio of testing to flight time requires 54140~hours of PPT ground testing before flight. This is equivalent to over 6 years of continuous thrusting in a thermal vacuum chamber. 









% Additions to program
\section{Additions to \BW\ program} \label{sec:BW1-additions}

It is important to note that this project does not intend to give a complete, final trajectory for \BW. In fact, this would be impossible given unknowns such as launch date and payload. Rather, this project developed a procedure for determining a good trajectory once those parameters are known as well as identifying issues that need to be considered when developing the trajectory. Perhaps more importantly for mission planning, the work described in this thesis provides an estimate for the transfer duration and fuel requirements, as presented in \autoref{tab:Results}. This data should inform future revision of the mission architecture. In particular, the radiation modelling suggests that the selected boundary of the van Allen belts is too conservative; the cruise phase could start at a much lower altitude, using the much more fuel-efficient PPTs. Additionally, repeated simulations suggested the spacecraft is only susceptible to being recaptured for the first two or three orbits of the Moon during the capture phase. Consequently the descent phase could also begin much earlier, once again saving many kilograms of ammonia propellant by using the more efficient PPTs. Of course, these two strategies would both increase the total mission time. That is a decision for the project management team.

The power modelling undertaken during this project revealed that \BW\ experiences a power surplus for most of the lunar transfer. While payload standby and communications power requirements are unknown, they are not expected to exceed this surplus. Consequently, the vehicle configuration could be modified based on these results.

The excess power may be exploited by increasing the PPT pulse frequency or installing additional thruster units. Simulation of the higher thrust resulting from these modifications indicates a higher lunar approach velocity complicating rendezvous and strong capture. These simulations did perhaps suffer from implementing orbital energy as the termination condition, thus sacrificing approach velocity for low altitude. If \BW\ is to be modified to exploit the surplus power available, future trajectory modelling would be well advised to implement velocity relative to the Moon as a termination constraint. Alternatively, removing extraneous solar panels would significantly decrease the launch mass of the vehicle, and also simplify thermal control. 

Furthermore, the power surplus suggests that the developmental focus of thruster research at Stuttgart on power efficiency is perhaps misplaced. Rather, fuel efficiency could be improved by focussing research on $I_{sp}$ (by improving exhaust velocity). More practically useful would be improving the thrust; this would not only simplify complicated maneouvres such as the lunar capture and course corrections, but it would also decrease total transfer time, currently about 3.5~years. Of course, power efficiency may be important for other mission profiles, but to date the TALOS and SIMPLEX thrusters are only intended for use on \BW.








% Additions to knowledge
\section{Additions to low thrust trajectory optimisation} \label{sec:Optimisation-additions}

This project has undertaken one of the highest fidelity low-thrust trajectory optimisations present in literature. It has  demonstrated the feasibility of such high fidelity optimisation using SOCS, subject to the advent of faster processing and greater computer storage. While computer hardware will eventually cover this gap, in the short term SOCS suffers from its direct shooting limitation. The field of trajectory optimisation would benefit greatly from modification of the SOCS algorithm to support multiple shooting that could be distributed over multiple processor cores.




% Research
\section{Conclusions of the research}

As repeatedly highlighted in literature, making an initial guess remains one of the biggest problems to trajectory optimisation. This is in part because low thrust trajectories are extremely sensitive to small changes, resulting in narrow basins of convergence surrounded by trajectories that do not achieve lunar capture. Thus finding lunar capture scenarios is an important step in optimising the trajectory of \BW.

Furthermore, some conclusions can be made about the thrust profile in these initial guesses. At the start of this project a thrust profile was implemented tangential to the spacecraft's position. This circularises the orbit as it rises. Based on the limited thrust profiles available in STK a new strategy of thrusting along the velocity vector was modelled. This profile preserves the orbital elements, in particular eccentricity. While this profile loses some delta-v due to gravitational drag, maintaining a low periapsis magnifies the thrust due to the Oberth effect, while raising the apoapsis allows greater exploitation of lunar gravitational assists. The thrust profile would be further improved under the thrust power constraints by focussing thrust over the periapsis, and coasting over apoapsis.

This strategy was ignored when the reduced complexity ascent phase was allowed to reach a fully optimised solution. In this scenario, the spacecraft uses an intermittent thrust-coast-thrust profile. While there is a possibility that this is due to power limitations, further simulations indicated that the batteries and solar panels are sized well enough to not require coasting phases, even during higher energy arcjet phases. Rather, this profile seems to be an artefact of the phase objective: raising the periapsis while using as little propellant as possible. Thus the optimisation does achieve its objective of escaping the van Allen belts very effectively. However, the overall mission objective of getting to the Moon using as little propellant as possible may benefit from additionally raising the apoapsis during this phase, by exploiting the Oberth effect to maximise the orbital energy obtained per unit of propellant expended.

Regarding the power consumption, it was revealed not to be a limiting factor on the trajectory for the planned satellite configuration. However, the discrepancy between the optimised ascent phase and the higher fidelity ascent phase revealed that power generation does increase significantly as the Earth-Moon system approaches periapsis on its orbit around the Sun. Furthermore, due to the thrust vector constraint the spacecraft attitude only has one degree of freedom (roll). Thus the power generation is also heavily dependent on the right ascension. Unfortunately, both launch date and right ascension are dependent on the launch, which is not under the control of the Institute for Space Systems.






% Future work
\section{Future work} \label{sec:Future-work}

There are quite a number of improvements that should be made if the development of \BW\ is continued. 

First are a number of modelling techniques that the author came across during the project, and would implement if he were starting over. These include the parameterised departure date and thrust duty cycle described in \autoref{sub:Thrust-parameterisation}. Additionally, the thrust profile could be modelled using Euler angles instead of the unit vector described in \autoref{sub:Applied-Thrust}. Not only is this expected to further smooth the solution space, it would supercede one of the constraint evaluations; specifically, the repeated calculations to ensure the control vector is  a unit vector would no longer be required.

Secondly, subsequent to the mission architecture and vehicle configuration recommendations in \autoref{sec:BW1-additions} more detail would be required in the power modelling. Payload and communications power budgets would be required, and the power degradation model described in \autoref{sub:Power-generation} should be implemented in software.

% get cruise phase to terminate at apoapsis for capture
% ascent phase not so important, since it keeps going in circles around the same focus anyway.
% cruise phase adjusts departure date for lunar resonances (thrust profile not as important)
% ascent phase just changes thrust profile, departure date not as important since it doesn't get near Moon (therefore individual phase optimisation departure date not important; combined scenario departure date should be dominated by cruise phase)

Colleagues working on the project have identified a number of lunar inclinations that reduce station-keeping costs throughout the science phase. \textcite{Zeile2010} recommend a final inclination of 70\degrees, having used an argument of periapsis of 43.3\degrees\ and a right ascension of 285\degrees, corresponding to the orbital elements in \autoref{tab:Phase-5-constraints-revised}. This would be easily achievable using minor course corrections during the arcjet-powered capture phase, or even during the lengthy descent phase with the PPTs. The recent study by \textcite[Figure 14a]{Gupta2011} reinforces this conclusion, showing that the \enquote{transition altitude}, that is, the altitude above which a sudden increase in orbital life time is observed, is lowest for a 70\degrees\ inclined orbit and provides a range of right ascensions that could be trialled.

\begin{table}[ht]
\caption{Revised Keplerian elements for the final orbit of \BW\ trajectory optimisation (end of descent phase)}
\label{tab:Phase-5-constraints-revised}
\begin{center}
\begin{tabular} {cc}\toprule
Parameter & Value\\\midrule
Semimajor axis, $a$ (m) & $1.8371\times 10^6$\\
Eccentricity, $e$ (-) & 0.0\\
Inclination, $i$ (rad) & 1.222\\
Argument of periapsis, $\omega$ (rad) & 0.7382 \\
Lunar Longitude of the Ascending Node, $\Omega$ (rad) & 4.9742 \\\bottomrule
\end{tabular}
\end{center}
\end{table}

To derive a truly optimal trajectory it would be necessary to optimise all of the phases simultaneously to determine inter-dependencies. This was attempted early in the project, but abandoned due to hardware and software limitations. Once a completed trajectory is determined, a Monte Carlo analysis should be performed to determine the sensitivity of the trajectory to small changes in thruster performance, imperfections in gravitational modelling, and other parameters.

Finally, the author had an abstract accepted for the 62nd International Astronautical Congress on alternate mission profiles. Unfortunately completion of this thesis took precedence, but this work remains of interest to the \BW\ team. Of particular interest is the possibility to pass through, or send a parasitic probe to, the Kordylewski clouds hypothesised to exist at the Earth-Moon Lagrange points L4 and L5. The trajectory model presented herein could easily be adapted to such alternative mission scenarios.

Furthermore for any practical mission it is important to examine failure scenarios. For traditional  missions with a finite number of impulses, there are a finite number of possible failures. Continuous thrust leads to an infinite number of failure scenarios, and although these may be categorised and evaluated, such extensive work is often worthy of a PhD by itself \parencite{Renk_thesis}.


