\markboth{Abstract}{Abstract} % set headers

\vfill

The University of Stuttgart is conducting a research program to build a succession of small satellites. The ultimate goal of this program is to build and launch a craft named \BW\ (after the federal state that Stuttgart is situated in, Baden-W\"{u}rttemberg) into lunar orbit, for eventual impact with the Moon. As with the majority of space missions, launch cost is a severely limiting factor so it is necessary to carefully plan the trajectory before launch, to ensure lunar capture and minimise the amount of fuel needed by the spacecraft.

This thesis outlines work conducted to find a robust fuel-optimal trajectory for \BW\ to reach the Moon. Several unique aspects of this craft require a novel approach to that optimisation. Firstly, the spacecraft uses a new low-cost propulsion system, severely limiting maneouvrability and accessibility of transfer trajectories. Secondly, to reduce the mass and complexity of moving parts, the solar panels are fixed to the body; consequently, the craft must rotate itself to point its solar panels towards the Sun to recharge. No thrusting can occur during this time. This magnifies the effect of the third design decision, which is to restrict the dry mass of the craft by giving it very little on-board power storage. After approximately an hour of accelerating it is expected to need to coast for several hours to recharge its batteries, resulting in a relatively high frequency stop-go-stop thrust profile.

Due to these constraints, the trajectory optimisation is one of the most complex ever attempted. Since the craft will be built and launched, many simplifications made in purely theoretical studies could not be utilised, such as neglecting the weaker forces acting on the spacecraft in cis-lunar space. The very low thrust results in very long transfer times, during which even small magnitude forces acting on the spacecraft can significantly perturb its trajectory. However, including these forces creates non-linearities in the equations of motion associated with spacecraft trajectories, limiting the optimisation methods that could be used, and increasing computational complexity.

Optimisation methods for low-thrust spacecraft trajectories have been the subject of much research, but most studies conclude that knowledge is still lacking in this area. Furthermore, many optimisation methods investigated in existing literature are incompatible with the intermittent thrust profile required by the \BW\ thrusters. For this reason it was necessary to thoroughly review available optimisation methods and determine which may be adapted to this scenario. The resulting optimisation method was applied to the \BW\ scenario to determine an efficient thrusting profile that will get the craft to the Moon.

It was found that very few established optimisation algorithms can support the number of variables required for such a complex, long duration trajectory. The Sparse Optimal Control Software (SOCS) marketed by The Boeing Corporation was used via an interface developed at the University of Stuttgart called the the Graphical Environment for Simulation and Optimisation (GESOP). Due to unknown constraints such as launch date, the phases defined by the mission architecture were modelled and optimised independently. This approach allows mission planning flexibility while still providing reliable estimates for optimal fuel use, mission duration and power limitations.

A trajectory is presented for each of the phases, ascending from the intial geosynchronous transfer orbit (GTO) to the eventual low lunar orbit (LLO). The resulting science phase is propagated forward in time to ensure orbital lifetime meets the mission requirements. Recommendations are subsequently made for the continuing development of the mission architecture.

The primary outcome of this study is a procedure for developing an operational trajectory for \BW\ after launch details are known. Given the current mission architecture and assumed launch details, the thermal arcjet requires 1205~hours (50.2~days) of operation while consuming 93~kg of ammonia propellant, and the pulsed plasma thrusters require 29177~hours (3.3~years) of operation while consuming 19~kg propellant. Power constraints were not found to be mission limiting for the current spacecraft configuration. Consequently, although the laboratory testing burden on the PPTs is already quite heavy, it is recommended that the mission architecture be adjusted to shorten arcjet phases and lengthen PPT phases. Furthermore, this project found that the optimisation package SOCS was the best commercially available option for low-thrust trajectory optimisation, but that it would benefit greatly by adaptation to a parallel shooting algorithm that may be distributed amongst multiple computer processors.

\vfill