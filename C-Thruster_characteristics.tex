\chapter{Thruster characteristics} \label{cha:Thruster-characteristics}

\section{Pulsed plasma thrusters}\label{sec:PPT-characteristics}

\autoref{tab:PPT-performance} shows data measured in the laboratory at the Institute for Space Systems at the University of Stuttgart, and provided to the author by Mr. Matthias Lau. All tests were performed at a pulse frequency of 1~Hz. Thrust and energy consumption may be scaled linearly by varying pulse frequency, subject to wear and tear limiting the total number of pulses that the PPTs can deliver, and the thermal equilibrium of the materials attenuating the heat generated by the thruster.

\begin{table}[h]
\caption{Performance of the instationary magnetoplasmadynamic thrusters from laboratory tests at IRS, University of Stuttgart}
\label{tab:PPT-performance}
\begin{center}
\begin{tabular}{C{0.1\textwidth} C{0.1\textwidth} C{0.1\textwidth} C{0.12\textwidth} C{0.12\textwidth} C{0.13\textwidth}}\toprule
Power (W) &  Mass bit ($\mu$g) & Thrust (mN) & Exhaust velocity (m/s) & Specific impulse (s) & Thrust efficiency (\%) \tabularnewline\midrule
% TEMP
4.3 & 50 & 0.12 & 2400 & 245 & 3.3 \tabularnewline
% SIMP-LEX (33.59$\mu$f)
66 & 49 & 0.89 &  18347 & 1870 & 12.4 \tabularnewline
% ADD-SIMP-LEX-3C (80$\mu$F)
67.6 & 53 & 1.373 & 25721 & 2622 & 26.1 \tabularnewline
% ASL-3D (60$\mu$F)
16.7 & 16 & 0.421 & 25844 & 2634 & 32.6 \tabularnewline\bottomrule
% SIMPLEX (first data from Matthias)
% 70 & 53 & 1.373 & 25000 & 2549 & \tabularnewline
% 55 & 45 & 1.219 & 27000 & 2753 & \tabularnewline
\end{tabular}
\end{center}
\end{table}

The first entry is from an early test bench model called TEMP. The second entry is the SIMPLEX PPT, operating with a capacitance of 33.59~$\mu$F to trigger the pulse. The third entry is for the modified ADD-SIMPLEX-3C, operating at 80~$\mu$F. The final entry is ASL-3D, at 60~$\mu$F.


\section{Thermal arcjet}\label{sec:Arcjet-characteristics}

The data in \autoref{tab:Arcjet-performance} was measured in the laboratory at the Institute for Space Systems at the University of Stuttgart, and was provided to the author by Mr. Birk Wollenhaupt. Four separate arcjets developed at the IRS were included in this data: VELARC, ATOS 1, ATOS 2 / ARTUR and TALOS. All four were tested using ammonia (NH$_3$) as the reaction mass. The ATOS 1 thruster was flight tested on board AMSAT-P-3D, launched by DLR in 2000.

\begin{table}[h]
\caption{Performance of thermal arcjet models from laboratory tests at IRS, University of Stuttgart}
\label{tab:Arcjet-performance}
\begin{center}
\begin{tabular}{c C{0.09\textwidth} C{0.12\textwidth} C{0.1\textwidth} C{0.12\textwidth} C{0.12\textwidth} C{0.13\textwidth} }\toprule
& Power (W) & Mass flow rate (mg/s) & Thrust (mN) & Exhaust velocity (m/s) & Specific impulse (s) & Thrust efficiency (\%) \tabularnewline\midrule
% VELARC \\
\multirow{5}{*}{\rotatebox{90}{VELARC}}
& 375 & 12.5 & 45 & 3630 & 370 & 22.0 \tabularnewline
& 280 & 9.3 & 35 & 3710 & 378 & 22.9 \tabularnewline
& 680 & 20.8 & 98 & 4688 & 478 & 33.6 \tabularnewline
& 260 & 6.7 & 26 & 3806 & 388 & 18.7 \tabularnewline
& 345 & 10.7 & 43 & 3995 & 407 & 24.8 \tabularnewline\midrule
% ATOS 1 \\
\multirow{3}{*}{\rotatebox{90}{ATOS 1}}
& 750 & 24 & 115 & 4750 & 484 & 36.7 \tabularnewline
& 748 & 24 & 114 & 4709 & 480 & 36.2 \tabularnewline
& 750 & 22.5 & 95 & 4222 & 430 & 26.7 \tabularnewline\midrule
% ATOS 2 ARTUR \\
\multirow{8}{*}{\rotatebox{90}{ATOS 2 / ARTUR}}
& 796 & 22.1 & 107 & 4836 & 493 & 32.5 \tabularnewline
& 1503 & 22.1 & 137 & 6200 & 632 & 28.3 \tabularnewline
& 892 & 26.8 & 128 & 4758 & 485 & 34.0 \tabularnewline
& 1512 & 26.8 & 161 & 6004 & 612 & 32.0 \tabularnewline
& 868 & 31 & 144 & 4630 & 472 & 38.3 \tabularnewline
& 1463 & 31 & 183 & 5915 & 603 & 37.1 \tabularnewline
& 983 & 36 & 167 & 4640 & 473 & 39.4 \tabularnewline
& 1858 & 36 & 221 & 6131 & 625 & 36.4 \tabularnewline\midrule
% TALOS \\
\multirow{7}{*}{\rotatebox{90}{TALOS}}
& 800 & 21.5 & 103 & 4767 & 486 & 30.5 \tabularnewline
& 796 & 21 & 111 & 5297 & 540 & 37.0 \tabularnewline
& 800 & 21.9 & 108 & 4915 & 501 & 33.1 \tabularnewline
& 800 & 20.0 & 103 & 5140 & 524 & 32.9 \tabularnewline
& 800 & 18.0 &  95 & 5278 & 538 & 31.3 \tabularnewline
& 800 & 15.9 & 85 & 5366 & 547 & 28.6 \tabularnewline
& 800 & 13.8 & 74 & 5356 & 546 & 24.8 \tabularnewline\bottomrule
\end{tabular}
\end{center}
\end{table}

