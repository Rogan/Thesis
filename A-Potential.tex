\chapter{Gravitational potential} \label{cha:Gravitational-potential}

\autoref{sub:Oblateness} introduces potential energy models, to accurately calculate the gravitational field around a central body, and thus the gravitational forces a nearby spacecraft would experience. The rather complicated equation \eqref{eq:A-Grav-Potential} is reproduced from \textcite{WGS84}, representing the gravitational component of the potential (compared to the angular momentum component),
\begin{equation} \label{eq:A-Grav-Potential}
V=\frac{\mu}{r}\left[1+\sum_{n=2}^{n_{max}}\sum_{m=0}^{n}\left(\frac{a}{r}\right)^{n}\bar{P}_{nm}(\sin\phi')(\bar{C}_{nm}\cos m\lambda+\bar{S}_{nm}\sin m\lambda)\right].
\end{equation}

The parameters $n$ and $m$ are the degree and order of the normalised gravitational coefficients, $r$ is the distance from the body's centre of mass, $\phi'$ is the polar latitude and $\lambda$ is the polar longitude. The gravitational coefficients, $\bar{C}_{nm}$ and $\bar{C}_{nm}$ are published in data tables with the gravitational constant for the central body, $\mu$, and the semimajor axis of the oblate body, $a$. For $m = 0$, $k = 1$ and for $m \neq 0$, $k = 2$.
The remaining term within the two summation series, is described as a normalised associated Legendre polynomial,
\begin{subequations}\label{eq:A-Legendre-polys}
\begin{equation}
\bar{P}_{nm}(\sin\phi') = \left[\frac{(n-m)!(2n+1)k}{(n+m)!}\right]^{\frac{1}{2}}P_{nm}(\sin\phi'),
\end{equation}
where $P_{nm}(\sin\phi')$ is the associated Legendre polynomial,
\begin{equation}
P_{nm}(\sin\phi') = (\cos\phi')^{m}\frac{d^{m}}{d(\sin\phi')^{m}}[P_{n}(\sin\phi')], 
\end{equation}
and $P_{n}\left(\sin\phi'\right)$ is the Legendre polynomial,
\begin{equation}
P_{n}(\sin\phi') = \frac{d^{n}}{2^{n}n!d(\sin\phi')^{n}}(\sin^{2}\phi'-1)^{n}.
\end{equation}
\end{subequations}

According to \textcite{WGS84} the series is theoretically valid for $r\geq a$, though it can be used with probably negligible error near or on the body's surface, that is, $r\geq\text{body's surface}$. The series should not be used for $r<\text{body's surface}$.

Due to minor differences in the definitions of gravitational harmonics, the zonal harmonic, $J_2$, is equivalent to $-C_{2,0}$.

% The normalised gravitational coefficients $\bar{C}_{nm}$ and $\bar{S}_{nm}$ for the Earth were taken from the Joint Gravity Model 3 \parencite{Tapley1996} due to availability of data. Several newer models have been released such as the Gravity Recovery and Climate Experiment \parencite{EIGEN-5C}, but the JGM3 accuracy was sufficient for this project. For similar reasons the lunar gravitational model was used from the Lunar Prospector mission \parencite[LP165, ][]{Konopliv2001}.