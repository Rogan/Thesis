\chapter{Gravitational potential} \label{cha:Gravitational-potential}

\autoref{sub:Oblateness} introduces gravitational potential models, to accurately calculate the gravitational field around a central body, and thus the gravitational forces a nearby spacecraft would experience. The rather complicated equation \eqref{eq:A-Grav-Potential} is reproduced from \textcite{WGS84}. 
\begin{equation} \label{eq:A-Grav-Potential}
V=\frac{\mu}{r}\left[1+\sum_{n=2}^{n_{max}}\sum_{m=0}^{n}\left(\frac{a}{r}\right)^{n}\bar{P}_{nm}\left(\sin\phi'\right)\left(\bar{C}_{nm}\cos m\lambda+\bar{S}_{nm}\sin m\lambda\right)\right]
\end{equation}

The parameters $n$ and $m$ are the degree and order of the normalised gravitational coefficients, $r$ is the distance from the body's centre of mass, $\phi'$ is the polar latitude and $\lambda$ is the polar longitude. For $m = 0$, $k = 1$ and for $m \neq 0$, $k = 2$.
The remaining term within the two summation series, is described as a normalised associated Legendre polynomial. \\

\begin{subequations}\label{eq:A-Legendre-polys}
\centering 
Normalized associated Legendre polynomial
\begin{equation}
\bar{P}_{nm}\left(\sin\phi'\right) = \left[\frac{\left(n-m\right)!\left(2n+1\right)k}{\left(n+m\right)!}\right]^{\frac{1}{2}}P_{nm}\left(\sin\phi'\right)
\end{equation}
Associated Legendre polynomial
\begin{equation}
P_{nm}\left(\sin\phi'\right) = \left(\cos\phi'\right)^{m}\frac{d^{m}}{d\left(\sin\phi'\right)^{m}}\left[P_{n}\left(\sin\phi'\right)\right]
\end{equation}
Legendre polynomial
\begin{equation}
P_{n}\left(\sin\phi'\right) = \frac{d^{n}}{2^{n}n!d\left(\sin\phi'\right)^{n}}\left(\sin^{2}\phi'-1\right)^{n}
\end{equation}
\end{subequations}

According to \textcite{WGS84} the series is theoretically valid for $r\geq a$, though it can be used with probably negligible error near or on the body's surface, that is, $r\geq\text{body's surface}$. The series should not be used for $r<\text{body's surface}$.

% The normalised gravitational coefficients $\bar{C}_{nm}$ and $\bar{S}_{nm}$ for the Earth were taken from the Joint Gravity Model 3 \parencite{Tapley1996} due to availability of data. Several newer models have been released such as the Gravity Recovery and Climate Experiment \parencite{EIGEN-5C}, but the JGM3 accuracy was sufficient for this project. For similar reasons the lunar gravitational model was used from the Lunar Prospector mission \parencite[LP165, ][]{Konopliv2001}.