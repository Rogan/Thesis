\chapter{Vehicle modelling and parameterisation}\label{cha:Vehicle}

\section{Vehicle modelling}

% Power
\subsection{Power}\label{sub:Vehicle-power}

modelled power as one of the states
battery capacity from \cite{Falke2004}
integrate power used over the transfer (active thruster * dt * dL/dt * dLn/dL)
panel efficiency from FLP @ Uryu
Power generated
\begin{gather}
P = E_{density} \times\eta_{panel}\times\eta_{area}\times\eta_{converter}\times\cos\theta_{sun} - P_{instruments} - P_{comms}
\end{gather}
sun angle = angle between sun vector and control vector
\begin{gather}
a\cdot b = |a|.|b|\cos{\theta} \\
\theta = \arccos\frac{a\cdot b}{|a|.|b|}
\end{gather}
Batteries 2.8kWh \parencite{Falke2004}
solar panel degradation is directly related to total equivalent \enquote{fluence} (i.e. radiation dose), figure is available
(\cite{Hechler2002} Power degradation modelling in the SMART 1 mission analysis, CNES Seminar on van Allen Radiation Modelization, ESOC/TOS-GMA)
\parencite{Erb_thesis}
10~sqm TEC-STAR GaAs/InP 34 series solar arrays with 200~$\mu$m front cover glass and 500~$\mu$m back cover glass
\begin{equation}
P(t_0) = P_{max} = 1.845\text{kW}
\end{equation}
must integrate the total dosage, where the derivative with respect to time dFdt is approximated as a function of the radius (2.4 per 24 hours - just multiply by deltaT) and initial dosage $\mathfrak{F}$ is zero.
\enquote{state} of fluence can then be looked up on another B-spline (fitted curve) to get relative power degradation, then 
\begin{equation}
P_{actual} = P_{max} \times \mathfrak{R}
\end{equation}
is supplied to the batteries. (need to allow for sun angle too!)
subtract power required for spacecraft bus
graph of unconstrained trajectory power output - explain need for thrust duty cycle parameter $\zeta$


regardless, it was found that increasing the thrust shortens the transfer but makes lunar capture more difficult. The higher thrust may be targeted at the Moon, and an appropriate orbital energy can be attained for lunar capture (about 5000~m$^2$s$^{-2}$) but due to the higher thrust, a larger proportion of the energy is made up of kinetic energy. In other words, the spacecraft is going too fast to achieve a strong lunar capture, even with a very low lunar pass. Perhaps relative velocity might be a better termination criterion?

% Propulsion
\subsection{Propulsion}
Arcjet power consumption \& operational Isp, efficiency, thrust based on Birk's work. \autoref{sec:PPT-characteristics}
PPTs similar from Matthias. \autoref{sec:Arcjet-characteristics}
science payload during science phase approx 300~W \parencite{web_BW-1}
\begin{gather}
I_{sp} = \frac{v_e}{g_0} \\
T = I_{sp}.g_0.\dot{m} = v_e.\dot{m} \\
\eta = \frac{T^2}{2\dot{m}P} \\
P = \frac{T^2}{2\dot{m}\eta} 
\end{gather}
modelled around apoapsis for first phase (objective is to raise periapsis outside VABs - apoapsis is already outside)

modelled around periapsis for second phase (objective is to raise orbital energy - Oberth effect means more delta V is gained by burning at low altitude).

optimisable fixed-body thrust vector allows trade-off between thrust angle and solar panel orientation - if the craft isn't thrusting, it can still use the thrust vector controls to point the solar panels at the sun


The length of this phase frequently resulted in computational difficulty, so a number of design compromises were investigated to solve the problem (it is acknowledged by the author that changing hardware design to make the simulation easier is a poor design philosophy, but the project management was interested to know the effects of these changes regardless of any simulation difficulties). For example, a design configuration was being investigated with 6 PPTs rather than 4, thereby increasing the available thrust without losing $I_{sp}$. The trade-off with this change is the additional thrusters would increase the dry mass of the vehicle, decreasing available payload. A similar design change was investigating the pulse rate of the PPTs. A higher pulse frequency increases thrust, increases mass flow, and increases power consumption, but impulse is conserved. However, an additional unknown is introduced: the heat flux within the electrodes is increased. 

can cause damage to electrodes, decrease operational lifetime - important on a 4 year mission! there is literature indicating that some designs can tolerate higher heat flux (cite Russian study) but further testing of the SIMPLEX would be required.


variable thrust magnitude - arcjet model mass flow rate + power?

PPTs is pulse frequency. Assuming we can run at 3~Hz, maximum thrust is increased.... change whole model?!?

Modelling with initial values of PPT frequency = 1~Hz, computational issues due to transfer being too long. Noticed that the power was a huge surplus, so spoke to the PPT experts (Matthias) who indicated that extra power could be pushed into higher pulse rate on the PPTs, up to 3~Hz. Remodelled based on this (effectively higher thrust and power consumption, but same Isp - this is the principle on which advocates of EP see it as the most efficient feasible propulsion for interplanetary missions).

Due to the particularly non-linear variations with launch date (periodic over 27.5~days based on the Moon's orbit) a good parameterisation of the launch date would be to use two variables, one to increment an integer number of months forwards or backwards, the other a floating point to indicate the number of days either side of the month.

Initially modelled Cruise with 4xPPTs at 1~Hz, took 2298~radians $\Delta$L and about 950~days. Optimisation not possible due to numerical limitations. Modified to 2Hz due to surplus power, simulation took 1347~radians and about 500~days.
\begin{figure}
\includegraphics[width=\textwidth]{Images/PPT_test.JPG}
\caption{SIMPLEX PPT during a laboratory test. The spark plug (centre left) ignites an arc between the two electrodes (pointing towards the camera) which vapourises the surface of the block of white PTFE between them. The electric field between the two electrodes then accelerates the plasma towards the camera.}\label{fig:PPT}
\end{figure}
\begin{figure}
\includegraphics[width=\textwidth]{Images/hiparc_betrieb.png}
\caption{TALOS arcjet during a laboratory test. An arc is generated between the axial cathode and the nozzle anode, which heats up the ammonia, accelerating it outwards.}\label{fig:arcjet}
\end{figure}
Show graphs for low power Cruise phases? 

From the power output, realised that the design could use more power to transfer faster - take many forms. Higher pulse frequency most promising, this linearly increases thrust at exactly the same Isp, at the cost of linearly increasing power consumption. The only limitation is wear and tear to the service life of the thruster; should be proportional to the number of pulses and since each pulse provides a certain Impulse bit, the number of pulses required to perform the lunar transfer should be constant. Thrusters have not been endurance tested yet meaning we don't know how many cycles they can perform. Other constraint is cooling - over about 3~Hz the PTFE is not able to cool properly between cycles and can interfere in the thruster performance. However, some tests performed in Russia (cite Matthias and/or Pedro here?) have successfully used a design similar to SIMPLEX at 20~Hz.

Thrust and power can also be linearly scaled by simply introducing more PPTs to the structure. This has been championed by the PPT designer, Matthias Lau. This option increases the redundancy of the system (and therefore reliability). However, it also carries the associated cost of extra weight.

Finally, extra power can be discharged into each pulse. This increases the Isp and thrust because the plasma is accelerated faster, but the increase is not proportional to the extra power put into each pulse. Therefore the power efficiency drops. Currently the development of the PPTs is focussed on improving the power efficiency, so the performance at this opposite end of the scale is not well characterised/not well known.

% Ground station
\section{Ground station access}
ground stations in stuttgart, USA, japan and australia give continual access during earth orbit

only downtime is when it's behind the moon

needs to rotate to communicate with ground station!!!

approx 250~W for Ka-band, 10~W for S-band. S-band only works in LEO.

% Earth shadow
\section{Earth shadow}

% Reaction wheel
\section{Reaction wheel desaturation}

% Parameterisation
\section{Parameterisation}

How everything is split up and discretised. Time, phases, gridpoints within phases. 

\subsection{Discretisation}

\subsection{Substitution of parameters}\label{sub:subst-param}

\autoref{sec:state-vector} establishes the state vector $\vec{x}=[p,f,g,h,k,L,m,t,E]$, and provides the differential equations with respect to time. However, as outlined in \autoref{sub:Independent-parameter}, the normalised true anomaly, $Ln$, was the independent parameter. Therefore a subsitution of parameters is required to give derivatives with respect to $Ln$, $\frac{d}{dLn}$. Differentiating \autoref{eq:Ln} gives
\begin{subequations}
\begin{gather}
\frac{dLn}{dt}=\frac{1}{\Delta L}\frac{dL}{dt} \\
\frac{dLn}{dt}\cdot\frac{dt}{dL}=\frac{1}{\Delta L} \\
\frac{dLn}{dL}=\frac{1}{\Delta L} \\
\frac{dL}{dLn}=\Delta L
\end{gather}
\end{subequations}

Using this identity, the state differential equations with respect to normalised longitude may be determined.

\begin{subequations}
\begin{align}
\frac{dp}{dLn}&=\frac{dp}{dt}\cdot\frac{dt}{dLn}\\
&=\frac{dp}{dt}\cdot\frac{dt}{dL}\cdot\frac{dL}{dLn}\\
&=\frac{\dot{p}}{\dot{L}}\Delta L
\end{align}
\end{subequations}

%The differential equations for $f$, $g$, $h$, $k$ and $m$ are modified similarly to $p$. $L$ is simply $\frac{dL}{dLn}$. Substituting the parameters of the remaining differential equations gives:
%\begin{subequations}
%\begin{align}
%\frac{dt}{dLn}&=\frac{dt}{dL}\cdot\frac{dL}{dLn}\\
%&= \frac{1}{\dot L}\cdot\Delta L
%\end{align}
%\end{subequations}
%\begin{subequations}
%\begin{align}
%\frac{dE}{dLn} &= \frac{dE}{dt}\cdot\frac{dt}{dL}\cdot\frac{dL}{dLn} \\
%&= \frac{P}{\dot{L}}\cdot\Delta L 
%\end{align}
%\end{subequations}
The remaining differential equations are modified similarly. The final equations are
\begin{subequations}
\begin{gather}
\frac{dp}{dLn}=\frac{\dot{p}}{\dot{L}}\Delta L \\
\frac{df}{dLn}=\frac{\dot{f}}{\dot{L}}\Delta L \\
\frac{dg}{dLn}=\frac{\dot{g}}{\dot{L}}\Delta L \\
\frac{dh}{dLn}=\frac{\dot{h}}{\dot{L}}\Delta L \\
\frac{dk}{dLn}=\frac{\dot{k}}{\dot{L}}\Delta L \\
\frac{dL}{dLn}=\Delta L \\
\frac{dm}{dLn}=\frac{\dot{m}}{\dot{L}}\Delta L \\
\frac{dt}{dLn}=\frac{1}{\dot{L}}\Delta L \\
\frac{dE}{dLn}=\frac{P}{\dot{L}}\Delta L 
\end{gather}
\end{subequations}
where $\dot{p}$, $\dot{f}$,  $\dot{g}$, $\dot{h}$, $\dot{k}$ and $\dot{L}$ are the original time-domain differential equations provided by \textcite{Walker1985}.

\subsubsection{Delta-v calculation}
The universal definition for delta-v was provided in \autoref{sub:Delta-v}. Unfortunately this definition also depends on time-domain integration. Therefore another substitution of parameters is required.
\begin{subequations}
\begin{align}
\Delta v &= \int^{Ln_f}_{Ln_i}\frac{|T|}{m}\frac{\text{dt}}{\text{dLn}}\text{dLn}\\
&= \int^{1}_{0}\frac{|T|}{m}\frac{\text{dt}}{\text{dL}}\cdot\frac{\text{dL}}{\text{dLn}}\text{dLn}\\
&= \int^{1}_{0}\frac{|T|}{m}\frac{\Delta L}{\dot L}\text{dLn}
\end{align}
\end{subequations}

%\section{Oberth effect}
%Thrust more at periapsis, less at apoapsis. Should become more apparent when thrust is not held to $T_{max}$.


% Lunar capture
Even with capture phases provided by arcjet, there is very little thrust available to perform lunar insertion. When simulating the transfer this problem is compounded by the fact that lunar insertion must be predicted very accurately, as simulating a lunar orbit in an Earth-centred frame, or vice-versa, causes the trajectory to periodically appear to move backwards relative to the central body. In other words, the anomaly decreases, which causes computational errors.

review this: Low Energy Motions in the Earth-Moon System, Chaos, and Weak Capture \parencite{Belbruno2007}

aim for a \emph{weak capture} at the Moon. This is a capture where the Kepler energy with respect to the Moon is non-positive and the motion of the particle with respect to the Moon is unstable. Such captures are generally temporary. Weak capture occurs in a special region in phase space about the Moon called the \emph{weak stability boundary}, WSB, rigorously defined by \textcite{Belbruno2004}

all Belbruno stuff is for impulsive transfers; WSB trajectory to the Moon through deep space.

Hiten probe tested aerobraking and WSB transfers. Also Kordylewski clouds.
% Discretisation
\cite{ASTOS_guide} \enquote{all major grid nodes are also control nodes by definition. When the major grid is too coarse for the controls, pure control nodes can be added.} \enquote{Collocation methods ignore any control refinement points.} \enquote{The constraint grid will be ignored [by] SOCS, since SOCS computes constraint evaluation only at each collocation grid point.}
% Parameterisation
% Vehicle

\section{Orbital behaviour}

\subsection{Gravitational assists} \label{sub:Grav-assist}

Due to the low thrust of spacecraft \BW, it is expected that a substantial part of the orbit raising will be achieved by exploiting gravitational assists from the Moon. This technique has been well studied, although previous higher-thrust missions did not find it efficient to attempt more than one or two lunar resonances. \textcite{Kemble2006} explains that \enquote{It is possible to utilise lunar gravity to assist in the orbit raising prior to lunar encounter. This takes the form of a \emph{gravitational pumping} effect if the correct phase with respect to the Moon can be established}. He goes on to quantify that \enquote{A typical $\Delta V$ saving of 800~ms$^{-1}$ is obtained by use of lunar gravity assist}. Due to the extended duration of \BW's transit, more lunar resonances should be possible than in a \enquote{typical} scenario, so this should reduce the required delta-v for \BW's ascent from GTO to lunar orbit from 4.1~kms$^{-1}$ to less than 3.3~kms$^{-1}$.

Because lunar gravitational assists provide delta-v without fuel expenditure, they should be implicitly accounted for during optimisation. Preliminary results have already shown this to be the case, as seen in \autoref{fig:Lunar-resonance}. From an initial orbit of 180,000~km (semi-cis-lunar, representing the later stages of the transfer to highlight the lunar assist) the spacecraft takes two orbits to align its phase with the Moon, then receives a very apparent boost purely from the Moon's gravity.

\begin{figure}
\centering
%\includegraphics{}
\caption{Preliminary optimisation showing a lunar resonance implicitly realised by the optimiser.}
\label{fig:Lunar-resonance}
\end{figure}

\subsection{Oberth effect} \label{sub:Oberth}

Regarding the thrust profile, it is well known that thrusting at particular points in the orbit are more efficient than others. \textcite{Kemble2006} once again explains, \enquote{Insertion of coast arcs allows the apogee raising input to be concentrated closer to perigee and hence increase the efficiency of the transfer (in terms of $\Delta V$ required)}. The reason for concentrating thrust near the perigee is known as the Oberth effect: over a period of thrusting, the specific orbital energy gained per unit delta-v exerted is equal to the instantaneous speed \parencite{Oberth1923}. Therefore thrusting is more efficient at high speed, which occurs at periapsis.
 
The resultant effect on the trajectory is that when variable thrust is implemented, the majority of the thrusting should occur near perigee; there should be an inverse relationship between thrust magnitude and orbital radius. 

\begin{figure}
\centering
%\input{Images/thrust-vs-time.jpg}
\caption{Thrust profile during orbit raising.}
\label{fig:Thrust_profile}
\end{figure}

Continuous thrust (tangential to the orbital radius) leads to circular trajectories because an eccentric starting trajectory spends more time at apoapsis than periapsis; hence more thrusting occurs at apoapsis than periapsis. As demonstrated by a Hohmann transfer, thrusting at apoapsis circularises the orbit.

The cycling observed in the figure is due to the elliptical orbit; at apogee and perigee the spacecraft is travelling perpendicularly to its position vector, resulting in a pitch angle of zero. As it travels towards the Earth, it has negative pitch angle, and positive pitch as it travels away again.
 
\begin{figure}
\centering
%\input{Images/pitch-vs-time.jpg}
\caption{Pitch angle during the orbit raising.}
\label{fig:BW1-Pitch}
\end{figure}

As observed in \autoref{fig:Thrust_profile} the thrust of \BW\ is almost constant throughout this simplified scenario. Consequently the traditional parameter for aircraft performance, thrust-to-weight ratio, seen in \autoref{fig:BW1-T/W-ratio}, is dominated by the changing effective weight, as the spacecraft moves closer to the Earth and then away again.

\begin{figure}
\centering
%\input{Images/thrust-to-weight-and-radius-vs-time.jpg}
\caption{\BW\ T/W ratio compared to radial position during the orbit raising. Plot exported from ASTOS.}
\label{fig:BW1-T/W-ratio}
\end{figure}

A spacecraft orbiting the Moon (in the same direction as the Moon orbits the Earth) undergoes a period of low velocity relative to the Earth every time it is on the Earthward side of the Moon. A spacecraft orbiting the Earth in a highly elliptical orbit undergoes a period of low velocity relative to the Earth when it reaches apoapsis, which is conveniently also the furthest point from the Earth, as shown in \autoref{fig:Apoapsis-and-periapsis}. Therefore the trajectory is expected to deliver the spacecraft at apoapsis to a \enquote{stationary point} in the intended lunar orbit, such that the Moon's gravity will then pull it into a steady lunar orbit, avoiding the need for a high-thrust capture maneuvre.

\begin{figure}
\centering
\includegraphics[scale=0.4]{Images/apsides.pdf}
\caption{Apoapsis and periapsis.}
\label{fig:Apoapsis-and-periapsis}
\end{figure}

To maximise the gravitational effect of the Moon on the spacecraft, it should be phase-locked with respect to the Moon's orbit. However, this would require constantly adjusting the orbital line of apsides, which is a high delta-v maneuvre. Gravitational resonance is generally exploited by pointing the apoapsis towards the Moon's ascending node (to remove any dependence on inclination) and adjust the orbital period to resonate with the Moon (that is, the spacecraft completes two orbits to every lunar orbit, or three orbits to every lunar orbit, or five orbits to every two lunar orbits, etcetera). However, based on the unique thrusting constraints of \BW, an alternative optimal scenario may be to maintain the fastest part of the orbit (periapsis) within the Earth's shadow (eclipse), to minimise the amount of time that the craft cannot charge its batteries. In this scenario the spacecraft should also thrust as it passes through the eclipse since it cannot recharge during this time, and thrusting will help it to escape the eclipse as quickly as possible. It will also be interesting to see whether it is optimal to correct inclination, apoapsis or eccentricity first, or whether the optimal solution adjusts all of these parameters simultaneously.
 
% It is well known how to most efficiently control various aspects of an orbit by thrusting at fortuitous times: increasing apoapsis efficiently by thrusting near periapsis, inclination by thrusting out of the orbital plane at apoapsis, and so forth, as anticipated by \textcite{Dachwald2007} in their discussion of optimisation results. \citet{Gao2008} outlines a convenient mathematical model to generate continuous, smooth parameters describing thrust steering, which can easily be used in an optimisation engine.

%\textcite{Edelbaum1964} gives an outline of some optimal low-thrust maneouvres (a, e, plane changes)
% arg of periapsis most efficiently changed at low eccentricity, inclination at high eccentricity, RAAN constant, a at periapsis.
 
% quantify stochastic & deterministic inputs, robustness