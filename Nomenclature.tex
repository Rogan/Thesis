\chapter*{Nomenclature}
\markboth{Nomenclature}{Nomenclature}

\section*{Notation}

Bold text represents a vector. A hat (for example $\hat{\vec{r}}$) represents a unit vector. A quantity that is normally a vector that is not in bold (for example $r$) represents the magnitude of that vector.

\begin{longtable}{l p{0.8\textwidth}}

\textbf{\autoref{cha:Rocket-science}} \\
$\epsilon$ & Specific orbital energy (m$^2$s$^{-2}$) \\
$\epsilon_k$ & Specific orbital kinetic energy (m$^2$s$^{-2}$) \\
$\epsilon_p$ & Specific orbital potential energy (m$^2$s$^{-2}$) \\
$\vec{v}$ & Velocity of spacecraft relative to central body (ms$^{-1}$ \\
$\mu$ & Gravitational constant of central body (m$^3$s$^{-2}$) \\
$\vec{r}$ & Distance of spacecraft from central body (m) \\
$I$ & Impulse (ms$^{-1}$) \\
$\vec{p}$ & Momentum (kgms$^{-1}$) \\
$I_{sp}$ & Specific impulse (s, see \autoref{sub:Isp}) \\
$g_0$ & Standard Earth gravity \parencite[9.80665~ms$^{-2}$,][]{CGPM} \\
$g(r)$ & Classic gravity relative to the primary body at $r$ metres from its centre \\
$m_{exhaust}$ & Mass of exhaust (kg) \\
$v_{exhaust}$, $v_e$ & Exhaust velocity (ms$^{-1}$) \\
$\Delta v$ & Delta-v (ms$^{-1}$, see \autoref{sub:Delta-v}) \\
$m$ & Mass of spacecraft (kg) \\
$\vec{T}$ & Applied thrust (N) \\
$\vec{D}$ & Aerodynamic drag (N) \\
$\gamma$ & Velocity vector angle (\degrees, see \autoref{fig:path-angles}) \\
$\alpha$ & Body axis angle (\degrees, see \autoref{fig:path-angles}) \\
$\varepsilon$ & Thrust angle (\degrees, see \autoref{fig:path-angles}) \\
$r_{SOI}$ & Radius of sphere of influence (m) \\
$a_s$ & Semimajor axis of the secondary body's orbit about the primary body (m) \\
$m_s$ & Mass of the secondary body (kg) \\
$m_p$ & Mass of the primary body (kg) \\
\\
\textbf{\autoref{cha:Orbital-dynamics-and-modelling}} \\
$\vec{r}$ & Position of spacecraft relative to primary body (m) \\
$\vec{v}$ & Velocity of spacecraft relative to primary body (ms$^{-1}$) \\
\\
$a$ & Keplerian element semimajor axis (m) \\
$e$ & Keplerian element eccentricity (-) \\
$i$ & Keplerian element inclination (\degrees) \\
$\omega$ & Keplerian element argument of periapsis (\degrees) \\
$\Omega$ & Keplerian element longitude of the ascending node (\degrees) \\
$\nu$ & Keplerian element true anomaly (\degrees) \\
\\
$p$ & Modified equinoctial element semilatus rectum (m) \\
$f$ & Modified equinoctial element f (-) \\
$g$ & Modified equinoctial element g (-) \\
$h$ & Modified equinoctial element h (-) \\
$k$ & Modified equinoctial element k (-) \\
$L$ & Modified equinoctial element true longitude (\degrees) \\
\\
$\vec{\hat{i}}_r$ & Unit vector in radial direction \\
$\vec{\hat{i}}_\theta$ & Unit vector tangential to primary body \\
$\vec{\hat{i}}_h$ & Unit vector in direction of orbital momentum \\
\\
$\Delta_r$ & Total force acting on spacecraft in the $\vec{\hat{i}}_r$ direction (N) \\
$\Delta_\theta$ & Total force acting on spacecraft in the $\vec{\hat{i}}_\theta$ direction (N) \\
$\Delta_h$ & Total force acting on spacecraft in the $\vec{\hat{i}}_h$ direction (N) \\
\\
$\vec{\Delta_q}$ & Total force on spacecraft due to third bodies (N) \\
$\vec{d}_j$ & Position of third body $j$ relative to spacecraft (m) \\
$\vec{s}_j$ & Position of third body $j$ relative to primary body (m) \\
\\
$\vec{\Delta_g}$ & Total force on spacecraft due to primary body oblateness (N) \\
$J_2$ & Second zonal harmonic coefficient of Earth\\
$J_3$ & Third zonal harmonic coefficient of Earth\\
$J_4$ & Fourth zonal harmonic coefficient of Earth\\
$W$ & Orbital energy (J) \\
$\Phi$ & Energy due to angular momentum of orbit (J) \\
$V$ & Gravitational potential energy of orbit (J) \\
$\bar{P}_{nm}\left(\sin\phi'\right)$ & Normalised associated Legendre polynomials\\
$C_{n,m}$ & Normalised gravitational coefficient \\
$S_{n,m}$ & Normalised gravitational coefficient \\
$r_{peri}$ & Periapsis of the orbit (m) \\
\\
$\vec{\Delta_\Sun}$ & Total force on spacecraft due to solar wind (N) \\
$\beta$ & optical reflection constant (-) \\
$A_{craft}$ & effective satellite projected area (m$^2$)\\
$r_\Sun$ & distance of satellite from centre of Sun (m) \\
%$a_\Sun$ & effective semimajor axis of satellite around Sun (m) \\
%$M_\Sun$ & mean anomaly of orbit around Sun (\degrees) \\
\\
$\vec{\Delta_T}$ & total force on spacecraft due to thrust (N) \\
$\vec{\hat{u}}$ & unit control vector governing thrust direction \\
\\
\textbf{\autoref{cha:Results}}
\\
$\Sun$ & Subscript relating parameter to the Sun \\
$\Earth$ & Subscript relating parameter to the Earth \\
$\Venus$ & Subscript relating parameter to the Venus \\
$\Mars$ & Subscript relating parameter to the Mars \\
$\Jupiter$ & Subscript relating parameter to the Jupiter 



\end{longtable}

\section*{Acronyms}

\begin{longtable}{l p{0.8\textwidth}}

AOCS & Attitude \& Orbit Control System \\
ASTOS & Aerospace Trajectory Optimisation Software \\
CAD & Computer Aided Design \\
CAMTOS & Collocation and Multiple Shooting Trajectory Optimisation Software \\
CGA & Constrained Genetic Algorithm \\
COTS & Commercial Off-The-Shelf \\
CNES & Centre National d'\'{E}tudes Spatiales \\
DLR & Deutsches Zentrum f\"{u}r Luft- und Raumfahrt \\
DSN & Deep Space Network \\
EADS & European Aeronautic Defence and Space Company \\
ECI & Earth Centred Inertial \\
ECR & Electron Cyclotron Resonance \\
EML & Earth-Moon Lagrange point \\
ESA & European Space Agency \\
ET & Ephemeris Time \\
ESTEC &  European Space Research and Technology Centre \\
GESOP & Graphical Environment for Simulation and Optimisation \\
GSLV & Geosynchronous Satellite Launch Vehicle \\
GEO & Geostationary (Earth) Orbit \\
GTO & Geosynchronous Transfer Orbit \\
HEO & High Earth Orbit \\
HLO & High Lunar Orbit \\
IAU & International Astronomical Union \\
ICRF & International Celestial Reference Frame \\
IEEE & Institute of Electrical \& Electronic Engineers \\
IERS & International Earth Rotation Service \\
IFR & Institut f\"{u}r Flugmekanik und Flugregelung \\
IRS & Institut f\"{u}r Raumfahrtsysteme \\
ISRO & Indian Space Research Organisation \\
ITRF & International Terrestrial Reference Frame \\
JAXA & Japanese Aerospace Exploration Agency \\
JD & Julian Date \\
JGM3 & Joint Gravity Model 3 \\
JPL & Jet Propulsion Laboratory \\
LEO & Low Earth Orbit \\
LLO & Low Lunar Orbit \\
LP165 & Lunar Prospector Gravity Model, degree and order 165 \\
NASA & National Aeronautics \& Space Administration \\
NIMA & National Imagery \& Mapping Agency \\
NLP & Non-Linear Programming \\
PPT & Pulsed Plasma Thruster \\
PROMIS & Parameterised tRajectory Optimisation by direct MultIple Shooting \\
PTFE & Polytetrafluoroethylene (Teflon\texttrademark) \\
SEL & Sun-Earth Lagrange point \\
SIMPLEX & Stuttgart Impulsing MagnetoPlasmadynamic thruster for Lunar EXploration \\
SNOPT & Sparse Nonlinear OPTimiser \\
SOCS & Sparse Optimal Control Software \\
SOI & Sphere of Influence \\
SQP & Sequential Quadratic Programming \\
SSO & Sun Synchronous Orbit \\
STK & Satellite Tool Kit \\
TALOS & Thermal Arcjet for Lunar Orbiting Satellite \\
TLI & Trans-lunar Injection \\
TROPIC & Trajectory OPtimisation by dIrect Collocation \\
TT & Terrestrial Time \\
UTC & Universal Coordinate Time


\end{longtable}